%
\documentclass[a4paper]{scrartcl}

\usepackage{amsmath, amssymb}
\usepackage{graphicx}
\usepackage{natbib}
\usepackage{hyperref}
\hypersetup{pdfpagemode = {UseNone},
            pdftitle = {On the integrals of ellipses in the r-phi plane},
            pdfauthor = {Thomas Rometsch},
            pdfsubject = {},
            pdfview = {FitH},
            pdfstartview = {FitH},
            colorlinks = {true},
            linkcolor = [rgb]{0,0.35,0.7},
            citecolor = [rgb]{0,0.35,0.7},
            filecolor = [rgb]{0.61,0,0},
            urlcolor = [rgb]{0.61,0,0},
           }
%%%%%%%%%%%%%%%%%%%%%%%%%%%%%%%%%%%%%%%%
\usepackage{txfonts}
%%%%%%%%%%%%%%%%%%%%%%%%%%%%%%%%%%%%%%%%
\usepackage{enumitem}
\usepackage{placeins}



% %%%%%%%%%%%%%%%%%%%%%%%%%
% %%% Commenting System %%%
% %%%%%%%%%%%%%%%%%%%%%%%%%

\usepackage{lineno}
\linenumbers


\usepackage[dvipsnames]{xcolor}



\begin{document}

\title{On the lifetime of vortices generated by gap-opening planets}

\author{Thomas Rometsch}

\date{\today}

\maketitle
%
%________________________________________________________________



\section{Ellipses in the $r-\phi$ plane}

During the vortex detection routine, $\varpi/\varpi_0$ is visualized and equi-vortensity
lines are compared to ellipses in the $r-\phi$ plane.

We define a vortex as the material inside the ellipse.

These ellispes are described by their center coorindates, $r_0$, $\phi_0$, and their extents
$2a_r$ and $2a_\phi$.

Thus the ellipse equation takes the form

\begin{align}\label{eqn:ellipse_equation}
  \left( \frac{r - r_0}{a_r} \right)^2 + \left( \frac{\phi - \phi_0}{a_\phi} \right)^2 = 1
\end{align}

Without loss of generality, $\phi_0 = 0$ by rotating the frame of reference.

Note, that the $\phi$ direction does not carry a unit of length.
The corresponding length is the distance on the arc at radius $r$.
This needs to be accounted for during integration of quantities over the ellipse.

The area of the ellipse in the $r-\phi$ plane does not carry physical meaning.
The physically relevant quantity is the area enclosed by the corresponding region in the disk.

In the $r$ - $s$ plane, where $s$ is the length on the arc along the azimuthal direction,
the ellipse in stretched for $r > r_0$ and pinched for $r < r_0$.

To calculate the area, we need to integrate in polar coordinates with the appropriate measure,
where $r$ and $\phi$ must fulfill Eq. \eqref{eqn:ellipse_equation}.
This is done by integrating from $r_0 - a_r$ to $r_0 + a_r$ in radial direction.
The azimuthal integration domain is $[-\alpha(r), \alpha(r)]$, where
$\alpha(r) = a_\phi \sqrt{1 - \left(\frac{r-r_0}{a_r}\right)^2}$ is defined by Eq. \eqref{eqn:ellipse_equation}.


\begin{align}
  A & = \int_{Ellipse} \mathrm{d}A
  = \int_{r_0 - a_r}^{r_0 + a_r} \int_{-\alpha(r)}^{\alpha(r)} r \mathrm{d}r \mathrm{d}\phi                \\
    & = 2 a_\phi \int_{r_0 - a_r}^{r_0 + a_r} \mathrm{d}r \, r \sqrt{1 - \left(\frac{r-r_0}{a_r}\right)^2} \\
    & = 2 a_r^2 a_\phi \int_{-1}^1 \mathrm{d}x \left(x + \frac{r_0}{a_r}\right) \sqrt{1-x^2}
  = 2 a_\phi a_r^2 \left( I_1 + \frac{r}{a_r} I_2 \right)
\end{align}

The substitution $r - r_0 = a_r x$ was and the integral was split in two parts.

The second integral disappears because the integrant is anti-symmetric:

\begin{align}
  I_2 = \int_{-1}^{1} \mathrm{d} x \sqrt{1 - x^2} = 0
\end{align}

The remaining integral can be identified as the area of a half-circle

\begin{align}
  I_2 = \int_{-1}^1 \mathrm{d} x \sqrt{1 - x^2} = \frac{\pi}{2}
\end{align}

Thus, the area in the $r - s$ plane appears to be equivalent to the usual formula with the extent $a_s = r_0 a_\phi$,
\begin{align}
  A = \pi a_r a_\phi r_0
\end{align}

A possible interpretation is that the stretching in the outer part is compensated by the pinching in the inner part.


To give a precise definition for the vortex, we fit bell curves to the surface density, $\Sigma$,
and the normalized vortensity, $\varpi/\varpi_0$.
The function is composed of two bell curves in radial and azimuthal direction as
\begin{align}
  f(r, \phi) = c + a \exp\left( - \frac{(r - r_0)^2}{\sigma_r^2} \right) \exp\left( - \frac{(\phi - \phi_0)^2}{\sigma_\phi^2} \right)\,.
\end{align}

Then the vortex can be defined as the region enclosed by an ellipse (in $r-\phi$)
with the extent is defined by the full-width-half-maximum of the bell curve, i.e. $a_r = \sqrt{2 \log(2)} \sigma_r$
and $a_\phi = \sqrt{2 \log(2)} \sigma_\phi$.
In general, any ratio $k = \frac{a}{\sigma}$ can be chosen.

Again, the physical relevant integrals need to be performed in the $r-s$ plane with limits defined by
the ellipse in the $r-\phi$ plane.
To evaluate the integral, we again rotate into a reference system such that $\phi_0 = 0$.
We will first split off the contribution from the constant part of $f$ by reusing the result for the area.

\begin{align}
  F & = \int_{Ellipse} f(r, \phi) \, \mathrm{d}A                                                \\
    & = \int_{r_0 - a_r}^{r_0 + a_r} \int_{-\alpha(r)}^{\alpha(r)} r \mathrm{d}r \mathrm{d}\phi
  f(r, \phi)
    & = A \,c + J
\end{align}

\begin{align}
  J & = a \int_{r_0 - a_r}^{r_0 + a_r} r \mathrm{d}r \exp\left( - \frac{(r-r_0)^2}{2 \sigma_r^2} \right)
  2 \int_0^{\alpha(r)} \mathrm{d}\phi \exp\left( - \frac{\phi^2}{2\sigma_\phi^2} \right)                 \\
\end{align}

The $\phi$ integral can be evaluated using the error function, $\mathrm{erf}$, and the substitution
$y = \frac{\phi}{\sqrt{2} \sigma_\phi}$
\begin{align}
   & \int_0^{\alpha(r)} \mathrm{d} \phi \exp\left( - \frac{\phi^2}{2\sigma_\phi^2} \right)
  = \sqrt{2} \sigma_\phi \int_0^{\frac{\alpha(r)}{\sqrt{2} \sigma_\phi}} \mathrm{d}y \exp(-y^2)                                            \\
   & = \sqrt{\frac{\pi}{2}} \sigma_\phi \mathrm{erf} \left( \frac{a_\phi}{\sqrt{2} \sigma_\phi} \sqrt{1 - \frac{(r-r_0)^2}{a_r^2}} \right)
\end{align}

Using this result and the substitution $r - r_0 = a_r x$

\begin{align}
  J & = \sqrt{2\pi} a \sigma_\phi \int_{r_0 - a_r}^{r_0 + a_r} r \mathrm{d}r \exp\left( - \frac{(r-r_0)^2}{2 \sigma_r^2}\right)
  \mathrm{erf} \left( \frac{k}{\sqrt{2}} \sqrt{1 - \frac{(r-r_0)^2}{a_r^2}} \right)                                             \\
    & = \sqrt{2\pi} a \sigma_\phi \sigma_r k \int_{-1}^{1} \mathrm{d}x (r_0 + a_r x)  \exp\left( - \frac{k^2}{2} x^2\right)
  \mathrm{erf} \left( \frac{k}{\sqrt{2}} \sqrt{1 - x^2} \right)                                                                 \\
    & = \sqrt{2\pi} a \sigma_\phi \sigma_r r_0 k \int_{-1}^{1} \mathrm{d}x \exp\left( - \frac{k^2}{2} x^2\right)
  \mathrm{erf} \left( \frac{k}{\sqrt{2}} \sqrt{1 - x^2} \right)                                                                 \\
    & = \sqrt{2\pi} a \sigma_\phi \sigma_r r_0 \mathcal{C}_k
\end{align}

The second term in the integrant of the second line, resulting from multiplication of $a_r x$ with the two functions,
is anti-symmetric and thus its integral vanishes.

At this point, the integral only depends on constants and can be integrated numerically to obtain $\mathcal{C}_k = k \int_{-1}^{1} \mathrm{d}x \exp\left( - \frac{k^2}{2} x^2\right)
  \mathrm{erf} \left( \frac{k}{\sqrt{2}} \sqrt{1 - x^2} \right)$.

Finally, the quantity integrated over the vortex region is
\begin{align}
  F & = c A + \sqrt{2\pi} a \sigma_\phi \sigma_r r_0 k \mathcal{C}_k                \\
    & = \sigma_\phi \sigma_r r_0 \left( c \pi + a \sqrt{2\pi} \mathcal{C}_k \right)
\end{align}

The stretching and pinching effects were cancelling each other.
Interestingly, and not to be expected, in the case of the double bell curve we can observe the same effect.

Some computed values for a half-width of one $\sigma$, the half-width at half maximum,
and two $\sigma$ are $\mathcal{C}_1 = 0.98628137356472$,
$\mathcal{C}_{2\log(2)} = 1.25331413731550$, and $\mathcal{C}_2 = 2.16739302711$.


\end{document}
